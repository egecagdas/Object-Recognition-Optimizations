%!TEX root = ../VorlageBA.tex
\chapter{Grundlagen}

\section{Digital Imagery}
Digital imagery refers to visual content in digital form, that can be recognized and displayed by computers. For the purposes of this paper, we need to make a distinction between the following image types.

\begin{itemize}
	\item{Visible Spectrum(RGB) Imagery}
	\item{Thermal(Infrared) Imagery}
\end{itemize}

\subsection{Visible Spectrum(RGB) Imagery}

\cite{nasa_visiblelight} defines the visible light spectrum as the part of the electromagnetic spectrum visible to the human eye, ranging from approximately 380 to 700 nanometers in wavelength. This range encompasses all the colors perceivable by the human eye, from violet to red.

\subsection{Thermal(Infrared, IR) Imagery}
Thermal imaging, or thermography, is the detection and measuring of radiation in the infrared spectrum being emitted from an object with the use of thermographic cameras. \citep{spi_thermal}

%TODO thermal imagery

%TODO machine learning

%TODO deep learning

	%TODO ML/DL optimizations

%TODO object detection

%TODO Object detection optimizations, feature selection, common algorithms and models etc.


%TODO örnek tezleri incele benzer grundlagen çıkar


